% - INFO GENERALES - %
% Pour que la compilation fonctionne sur Overleaf, il faut que le compilateur soit XeLaTeX (changeable dans le menu en haut à gauche) %
% ---------- Initialisation du document ---------- %
\documentclass[a4paper, 12pt]{report}
% - Indique que le document est en français - %
\usepackage[french]{babel}
% - Réglage des marges - %
\usepackage{geometry}
\geometry{hmargin=2.5cm,vmargin=1.5cm}
% - Utilisation des polices latines - %
\usepackage{lmodern}
\usepackage{fontspec}
\setmainfont{Arial}
% - Permet la bonne rentrée des accents et caractères spéciaux - %
\usepackage[utf8]{inputenc}
\usepackage[T1]{fontenc}
% - Utilisation de couleurs - %
\usepackage{xcolor}
% - Somaire cliquable - %
\usepackage{hyperref}
% - Outils pour mise en place de bibliographie - %
\usepackage{biblatex}
\addbibresource{biblio.bib}
% - Insertion des images et des légendes - %
\usepackage{graphicx} 
\usepackage{subcaption}
% - Code source jolis - %
\usepackage{minted}
% - Facilité d'écriture des équations mathématiques - %
\usepackage{amsmath,amsfonts,amssymb}
% - Améliore les interfaces des objets flottants - %
\usepackage{float}
\usepackage{csquotes}
\usepackage{textcomp}

% --------- Début du Rapport --------- %
\begin{document}
% Suppréssion border rouge tables
\hypersetup{pdfborder=0 0 0}
% - Numérotation des pages - %
\pagenumbering{arabic}
\setmainfont{Arial}

% ---------- Page de Garde ---------- %
\begin{center}
    
    \includegraphics[width=0.6\textwidth]{img/logo/UBFC.png}
    \Huge
    \vspace{2cm}
    \textbf{\\\vspace{0.4cm}Polytech Dijon\\ Filière — Année (1A, 2A, ...)}
    \vspace{0.75cm}
    \\MATIÈRE - TYPE

    \vspace{0.5cm}
    \definecolor{polytech}{cmyk}{1, 0.32, 0, 0.12}
    \color{polytech}\rule{\textwidth}{1pt}
    \begin{center}
        \vspace{0.3cm}
        \Huge
        \color{black}\textbf{INTITULE DU PROJET}   
    \end{center}
    \rule{\textwidth}{1pt}

    \vspace{0.5cm}
    \color{black}\Large 
    Auteurs :\\
    NOM Prénom\\
    NOM Prénom\\
    \vspace{0.5cm}

    \includegraphics[width=0.6\textwidth]{img/logo/Polytech_Dijon.png}
        
    \vfill
    \Large
    2024-2025
            
\end{center}
\newpage
% ---------- Table des matières ---------- %
\begingroup
    \renewcommand{\contentsname}{Sommaire}
    \tableofcontents
% ---------- Table des figures ---------- %
    \renewcommand{\listfigurename}{Table des Figures}
    \listoffigures
\endgroup
\newpage


% ---------- Compte Rendu  ---------- %
\chapter{Introduction}
%\input{intro}

\chapter{Conclusion Générale}
%\input{conclusion}

%\printbibliography[heading=bibintoc]

\end{document}

% --------- Aide Syntaxe  ---------- %

% Site pour implémenter facilement des formules mathématiques: https://editor.codecogs.com

% - Types de Titres - %
\section{Titre}
\subsection{Titre}
\subsubsection{Titre}

% - Image - %
\begin{figure}[H]
\centering
\includegraphics[width=0.7\textwidth]{img/.png}
\caption{TITRE Figure}
\label{fig:1}
\end{figure}

% - Image groupée - %
\begin{figure}[H]
\centering
\begin{subfigure}[b]{0.45\textwidth}
    \includegraphics[width=\textwidth]{img/.png}
\end{subfigure}
\begin{subfigure}[b]{0.45\textwidth}
    \includegraphics[width=\textwidth]{img/.png}
\end{subfigure}
\caption{TITRE Figure}
\label{fig:1}
\end{figure}

% - Tableau - %
\begin{figure}[H]
    \centering
    \begin{tabular}{|c|c|c|c|c|c|c|}
    \hline
    Ligne 1 & a & b & c & d & e & f \\
    \hline
    Autre ligne 1 & & & & & & \\
    \hline
    Autre ligne 2 & & & & & & \\
    \hline
    \end{tabular}
    \caption{TITRE Tableau}
    \label{fig:1} 
\end{figure}

% - Faire référence à une figure - %
\ref{fig:1}

% - Formule Mathématique - %
\begin{equation}
    \label{eq:1}
    a^2 + b^2 = c^2
\end{equation}


% - Code source avec minted - %
% Baground color : deactive by default / breaklines : active by default for long lines
\begin{minted}[frame=lines,
    framesep=2mm,
    baselinestretch=1.2,
    %bgcolor = lightgray,
    fontsize=\footnotesize,
    linenos,
    breaklines
    ]{java}
    public class HelloWorld 
    { 
        public static void main(String args[]) 
        { 
            System.out.println("Hello, World"); 
        } 
    }
\end{minted}